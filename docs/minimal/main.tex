\documentclass[10pt]{article}
\usepackage{notestex}

\begin{document}
\title{FLTeX Minimal Start}
\author{Your Name}
\affiliation{
Creator at FLibrary\\
Undergraduate Student at the University of \LaTeX\\
}
\emailAdd{YOUREMAIL@example.org}
\maketitle
\newpage
\pagestyle{fancynotes}

\section{Introduction}
This document is a guide on the minimal start for the template.

To starting use the template, start your \texttt{main.tex} with the following:
\begin{verbatim}
        \documentclass[10pt]{article}
        \usepackage{path/to/notestex}
        
        \begin{document}
        \title{FLTeX Minimal Start}
        \author{Your Name}
        \affiliation{
        Creator at FLibrary\\
        Undergraduate Student at the University of \LaTeX\\
        }
        \emailAdd{YOUREMAIL@example.org}
        \maketitle
        \newpage
        \pagestyle{fancynotes}
        
        \end{document}
\end{verbatim}

A few things to notice:
\begin{enumerate}
\item You shall modify the author, affiliation, and other fields accordingly
\item \textbf{replace \texttt{path/to/notestex} with the actual path!}
\item \textbf{Use Lua\LaTeX or Xe\LaTeX to compile your document.}
\end{enumerate}

If you would like to write Chinese, you may use the following as your \texttt{main.tex}:
\marginnote{\footnotesize Due to the fact that this guide is written in an English environment, we didn't use Chinese in the code snippet. However, it can.}

\begin{verbatim}
        \documentclass[10pt]{ctex}
        \usepackage{path/to/notestex}
        
        \begin{document}
        \title{FLTeX Minimal Start}
        \author{Your Chinese Name}
        \affiliation{
        Creator at FLibrary\\
        Undergraduate Student at the University of \LaTeX\\
        }
        \emailAdd{YOUREMAIL@example.org}
        \maketitle
        \newpage
        \pagestyle{fancynotes}
        
        \end{document}
\end{verbatim}

\end{document}
